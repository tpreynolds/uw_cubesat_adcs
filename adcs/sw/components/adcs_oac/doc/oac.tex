\documentclass[12pt]{article}

% Load commands & pakcages
\usepackage{graphicx}
%	\newcommand*{\Scale}[2][4]{\scalebox{#1}{$#2$}}%
\graphicspath{{../slides/images/}{../figs/}}
\usepackage{wrapfig}
\usepackage[margin = 1in]{geometry}
%\usepackage[caption=false,font=footnotesize]{subfig}
\usepackage{array}
\usepackage{setspace} 
\usepackage{lastpage}
\usepackage{fancyhdr}
 %\usepackage{subfigure}% subcaptions for subfigures
\usepackage{amsmath,amssymb,amsthm}
\usepackage{bm}
\usepackage{bbm}
\usepackage{csquotes}

\usepackage{epstopdf}
\usepackage{hyperref}
\usepackage{color}
\usepackage{float}
\usepackage{media9}
\usepackage{adjustbox}
\usepackage{multimedia}
\usepackage{subfig}
\usepackage{algorithm2e}
\usepackage{color}
\usepackage{mathtools}
\definecolor{brickred}{rgb}{0.8, 0.25, 0.33}
\definecolor{alizarin}{rgb}{0.82, 0.1, 0.26}

% Tikz
\usepackage{tikz}
\usetikzlibrary{calc,patterns,decorations.pathmorphing,decorations.markings}
\usepackage{pgfplots}
%\pgfplotsset{compat=1.12}
\usetikzlibrary{patterns}
\usetikzlibrary{math}

\usepackage[backend=bibtex,style=ieee]{biblatex}
\bibliography{PDG.bib}
% Number/Set stuff
\newcommand{\real}{{\mathbb{R}}}
\newcommand{\realp}{\real_+}
\newcommand{\realpp}{\real_{++}}
\newcommand{\complex}{{\mathbb{C}}}
\newcommand{\field}{{\mathbb{F}}}
\newcommand{\realpart}{\mathrm{Re}}
\newcommand{\impart}{\mathrm{Im}}
\newcommand{\psd}[1]{\mathbb{S}^{#1}_+}
\newcommand{\pd}[1]{\mathbb{S^{#1}}_{++}}
\newcommand{\definedas}{\coloneqq}
\newcommand{\cone}{\mathcal{K}}
\newcommand{\lcone}[1]{\real_+^{#1}}
\newcommand{\qcone}[1]{\mathcal{Q}^{#1}}
\newcommand{\conleq}{\preceq_{\cone}}
\newcommand{\congeq}{\succeq_{\cone}}

% Derivatives & Optimization
\newcommand{\argmin}{\operatorname{argmin}}
\newcommand{\der}[2]{\frac{d #1}{d #2}}
\newcommand{\pder}[2]{\frac{\partial #1}{\partial #2}}
\newcommand{\sproc}{$\mathcal{S}$-procedure }
\newcommand{\scvx}{\texttt{SCvx}}

% Coordinate frames
\newcommand{\inertial}{\mathcal{I}}
\newcommand{\body}{\mathcal{B}}
\newcommand{\landing}{\mathcal{O}}
\newcommand{\cframe}[2]{\mathcal{#1}_{#2}}
\newcommand{\Fbody}{\cframe{F}{\body}}
\newcommand{\Finertial}{\cframe{F}{\inertial}}
\newcommand{\DCM}[2]{C_{{#1} \rightarrow {#2}}}
\newcommand{\cItoB}{C_{\body \leftarrow \inertial}}
\newcommand{\cBtoI}{C_{\inertial \leftarrow \body}}
\newcommand{\cBO}{C_{\landing/\body}}
\newcommand{\cOB}{C_{\body/landing}}
\newcommand{\xI}{\bm{x}_{\inertial}}
\newcommand{\yI}{\bm{y}_{\inertial}}
\newcommand{\zI}{\bm{z}_{\inertial}}
\newcommand{\xB}{\bm{x}_{\body}}
\newcommand{\yB}{\bm{y}_{\body}}
\newcommand{\zB}{\bm{z}_{\body}}

% Time 
\newcommand{\tf}{t_f}
\newcommand{\ti}{t_0}
\newcommand{\tfmin}{t_{f,\min}}
\newcommand{\tfmax}{t_{f,\max}}

% States
\newcommand{\Hq}{H_{\q}}
\newcommand{\Hw}{H_{\omega}}

% General math
\newcommand{\dg}{^{\circ}}
\newcommand{\zeros}[2]{0_{#1 \times #2}}
\newcommand{\eye}[1]{I_{#1}}
\newcommand{\blkdiag}{\mathop{\bf blkdiag}}
\newcommand{\spec}[1]{\mathop{\bf spec}\left\lbrace #1 \right\rbrace}
\newcommand{\diag}{\mathop{\bf diag}}
\newcommand{\eig}[1]{\mathop{\bf eig}\left\lbrace #1 \right\rbrace}

% Rigid bodies
%\newcommand{\skew}[1]{{#1}^{\times}}
\newcommand{\gI}{\bm{g}_{\inertial}}
\newcommand{\gB}{\bm{g}_{\body}}
\newcommand{\JB}{J_{\body}}
\newcommand{\FI}{\bm{F}_{\inertial}}
\newcommand{\FB}{\bm{F}_{\body}}
\newcommand{\rI}{\bm{r}_{\inertial}}
\newcommand{\rIb}{\bar{\bm{r}}_{\inertial}}
\newcommand{\rB}{\bm{r}_{\body}}
\newcommand{\rBb}{\bar{\bm{r}}_{\body}}
\newcommand{\vI}{\bm{v}_{\inertial}}
\newcommand{\vB}{\bm{v}_{\body}}
\newcommand{\vIb}{\bar{\bm{v}}_{\inertial}}
\newcommand{\vBb}{\bar{\bm{v}}_{\body}}
\newcommand{\wB}{\bm{\omega}_{\body}}
\newcommand{\wBb}{\bar{\bm{\omega}}_{\body}}
\newcommand{\TB}{\bm{T}_{\body}}
\newcommand{\TBb}{\bar{\bm{T}}_{\body}}
\newcommand{\uB}{\bm{u}_{\body}}
\newcommand{\uI}{\bm{u}_{\inertial}}
\newcommand{\xb}{\bar{\bm{x}}}
\newcommand{\ub}{\bar{\bm{u}}}
\newcommand{\uBb}{\bar{\bm{u}}_{\body}}
\newcommand{\tauI}{\bm{\tau}_{\inertial}}
\newcommand{\tauIb}{\bar{\bm{\tau}}_{\inertial}}
\newcommand{\tauB}{\bm{\tau}_{\body}}
\newcommand{\tauBb}{\bar{\bm{\tau}}_{\body}}

\newcommand{\mi}{m_{\text{dry}}}
\newcommand{\mf}{m_{\text{wet}}}
\newcommand{\ri}[1]{\bm{r}_{#1,0}}
\newcommand{\vi}[1]{\bm{v}_{#1,0}}
\newcommand{\wi}{\bm{\omega}_{\body,i}}
\newcommand{\Ti}[1]{\bm{T}_{#1,i}}
\newcommand{\mk}{m_{k}}
\newcommand{\rk}[1]{\bm{r}_{#1,k}}
\newcommand{\vk}[1]{\bm{v}_{#1,k}}
\newcommand{\wk}{\bm{\omega}_{\body,k}}
\newcommand{\Tk}[1]{\bm{T}_{#1,k}}
\newcommand{\uk}[1]{\bm{u}_{#1,k}}
\newcommand{\mN}{m_{f}}
\newcommand{\rf}[1]{\bm{r}_{#1,f}}
\newcommand{\vf}[1]{\bm{v}_{#1,f}}
\newcommand{\wf}{\bm{\omega}_{\body,N}}
\newcommand{\Tf}[1]{\bm{T}_{#1,f}}
\newcommand{\uf}[1]{\bm{u}_{#1,f}}

\newcommand{\mdot}{\dot{m}}
\newcommand{\rdot}{\dot{\bm{r}}_{\inertial}}
\newcommand{\vdot}{\dot{\bm{v}}_{\inertial}}
\newcommand{\wdot}{\dot{\bm{\omega}}_{\body}}

% constraints
\newcommand{\vmax}{v_{\max}}
\newcommand{\wmax}{\omega_{\max}}
\newcommand{\gimbal}{\delta_{\max}}
\newcommand{\los}{\xi}
\newcommand{\losmax}{\los_{\max}}
\newcommand{\losvec}{\bm{p}_{\body}}
\newcommand{\rLoSI}{\bm{r}_{d,\inertial}}
\newcommand{\rLoSB}{\bm{r}_{d,\body}}
\newcommand{\gs}{\gamma}
\newcommand{\gsmax}{\gamma_{\max}}
\newcommand{\gsvec}{\zI}
\newcommand{\tiltmax}{\theta_{\max}}
\newcommand{\tilt}{\theta}
\newcommand{\Tmin}{T_{\min}}
\newcommand{\Tmax}{T_{\max}}
\newcommand{\umin}{u_{\min}}
\newcommand{\umax}{u_{\max}}
\newcommand{\dTmax}{\dot{T}_{\max}}
\newcommand{\dumax}{\dot{u}_{\max}}
\newcommand{\ME}{\tilde{M}_e}
\newcommand{\MI}{\tilde{M}_i}

% Quaternions & dual quaternions
\newcommand{\q}{\bm{q}}
\newcommand{\qb}{\bar{\q}}
\newcommand{\qi}[1]{\q_{i}}
\newcommand{\qdot}{\dot{\q}}
\newcommand{\dq}[1]{\tilde{\bm{#1}}}
\newcommand{\dqgB}{\dq{\bm{g}}_{\body}}
\newcommand{\dqgI}{\dq{\bm{g}}_{\inertial}}
\newcommand{\dqq}{\dq{q}}
\newcommand{\dqb}{\bar{\dqq}}
\newcommand{\dqw}{\dq{\omega}}
\newcommand{\dqx}{\dq{x}}
\newcommand{\dqu}{\dq{u}}
\newcommand{\dqdot}[1]{\dot{\tilde{\bm{#1}}}}
\newcommand{\qskew}[1]{[ #1 ]_{\otimes}}
\newcommand{\qskewstar}[1]{[ #1 ]_{\otimes}^*}
\newcommand{\qcross}[1]{ [ #1 ]_{\oslash}}
\newcommand{\qcrossstar}[1]{ [ #1 ]_{\oslash}^*}
\newcommand{\It}{\tilde{\bm{I}}_8}

% Optimization & SCvx
\newcommand{\Jthree}{J_{3-\text{DoF}}}
\newcommand{\Jsix}{J_{6-\text{DoF}}}
\newcommand{\Jpp}{J_{\text{pinpoint}}}
\newcommand{\Jsoft}{J_{\text{soft-land}}}
\newcommand{\Jminerr}[1]{J_{\text{min-error},#1}}
\newcommand{\xxo}{\bar{\bm{x}}}
\newcommand{\uuo}{\bar{\bm{u}}}
\newcommand{\sso}{\bar{\sigma}}
\newcommand{\wv}{w_{\nu}}

%% theorems and stuff
\newtheorem{theorem}{Theorem}[section]
\newtheorem{definition}{Definition}
\newtheorem{assumption}[theorem]{Assumption}
\newtheorem{lemma}[theorem]{Lemma}
\newtheorem{claim}[theorem]{Claim}
\newtheorem{remarks}[theorem]{Remarks}
\newtheorem{remark}[theorem]{Remark}
\newtheorem{example}[theorem]{Example}
\newtheorem{corollary}[theorem]{Corollary}
\newtheorem{proposition}{Proposition}
\newtheorem{constraint}{Constraint}
\newtheorem{problem}{Problem}

% Other
\newcommand{\eqnref}[1]{(\ref{#1})}
\newcommand{\class}[1]{\texttt{#1}}
\newcommand{\package}[1]{\texttt{#1}}
\newcommand{\file}[1]{\texttt{#1}}
\newcommand{\BibTeX}{\textsc{Bib}\TeX}
\definecolor{commentgreen}{RGB}{0, 153, 0}
\newcommand{\todo}[1]{{\color{blue}(TODO: #1)}}
\newcommand{\comment}[1]{{\color{commentgreen}(#1)}}
\newcommand{\code}[1]{\texttt{#1}}
%\input{mikz.tex}
\newcommand{\ttt}[1]{\texttt{#1}}

\addbibresource{PDG.bib}

\title{SOAC: SOC-i Optimal Attitude Controller}

 \author{
  Taylor P. Reynolds%
  \thanks{Doctoral Student, W.E. Boeing Department of Aeronautics \& Astronautics: \texttt{tpr6@uw.edu} } 
\\
  {\normalsize\itshape
  RAIN Lab -- University of Washington}
 }
 \date{\today}

\begin{document}
 
\maketitle

\section{Convex Optimization}

\subsection{Convex Cones}

Each proper cone $\cone$ defines a \textit{generalized inequality} such that we may write $G \bm{z} \conleq h$ if and only if $h - G \bm{z} \in \cone$. In particular we shall care about two cones, the \textit{linear cones} and the \textit{quadratic cones}. Linear cones are denoted by the set 
\begin{equation}
\lcone{n} = \{ \bm{z}\in\real^n \,|\, \bm{z} \geq 0 \},
\label{eq:lcone_set}
\end{equation}
where the inequality is understood pointwise. The linear cones are the nonnegative orthants in $\real^n$. The second cone that we will deal with are the second order cones, each of which are a set $\qcone{n+1}$ defined by
\begin{equation}
\qcone{n+1} = \big\{ (z_0,\bm{z}_n)\in\real\times\real^n \,|\, z_0 \geq \| \bm{z}_n \|_2 \big\}.
\label{eq:soc_set}
\end{equation}


To see how these are used to express various constraints as second order cone constraints, first note that a standard quadratic form is expressed as
\begin{equation}
\bm{x}^T A ^T A \bm{x} + b^T \bm{x} + c \leq 0
\label{eq:soc_quad}
\end{equation}
where $(A,b,c) \in \real^{n\times n} \times \real^n \times \real$ are constant constraint parameters. In standard form, this constraint is expressed by
\begin{equation}
\biggr\| \begin{bmatrix}
A \bm{x} \\ \frac{1}{\sqrt{2}}( 1 + c + b^T \bm{x} )
\end{bmatrix} \biggr\|_2 \leq \frac{1}{\sqrt{2}} ( 1 - b^T \bm{x} - c),
\label{eq:soc_std}
\end{equation} 
The proof of which follows from some straight-forward algebra. Now observe that according to the definition in~\eqref{eq:soc_set} we would like
\begin{equation*}
\begin{bmatrix}
\frac{1}{\sqrt{2}} ( 1 - b^T \bm{x} - c) \\
A \bm{x} \\
\frac{1}{\sqrt{2}} ( 1 + b^T \bm{x} + c)
\end{bmatrix} \in \qcone{n+2}
\end{equation*}
It should be clear that the first entry in the vector about is equivalent to $z_0$ in~\eqref{eq:soc_set}, and the remaining rows are equivalent to $\bm{z}_n$. Now, ECOS would like the matrix-vector pair $(G,h)$ such that $h-G\bm{x}\in\qcone{n+2}$, and so we set
\begin{equation}
h \definedas \begin{bmatrix}
\frac{1}{\sqrt{2}}( 1 - c ) \\ \zeros{n}{1} \\ \frac{1}{\sqrt{2}}(1+c)
\end{bmatrix} \quad G \definedas \begin{bmatrix}
\frac{b^T}{\sqrt{2}} \\ -A \\ -\frac{b^T}{\sqrt{2}}.
\end{bmatrix}
\end{equation}
Now, the pair $(G,h)$ may be passed to ECOS to represent the quadratic constraint~\eqref{eq:soc_quad}.



\section{Problem Formulation}

The continuous time optimal control problem to be solved is as follows. 
\begin{problem}\label{prob:prob1}
Find the time $\tf\in(\ti,\tfmax]$ and torque commands $\uB : (\ti,\tfmax] \rightarrow \real^3$ such that
\begin{subequations}
\begin{align}
\min_{\tf,\uB(\cdot)} &\quad \tf \\
\text{subject to} &\quad \qdot = \frac{1}{2} \q \otimes \wB\\
&\quad J \wdot = \uB - \wB^{\times} J \wB \\
&\quad \| \wB \|_{\infty} \leq \wmax \\
&\quad \| \uB \|_{\infty} \leq \umax \\
&\quad \q^T \ME \q < 2 \\
&\quad \q^T \MI \q < 2 \\
&\quad \tfmin \leq \tf \leq \tfmax \\
&\quad \q(\ti) = \q_0,\, \q(\tf) = \q_f,\,\wB(\ti) = \wB(\tf) = 0.
\end{align}
\end{subequations}
\end{problem} 

\paragraph{Linearization}
We refer to our \textit{state} as $\bm{x} = \big[ \q \; \wB \big]^T \in \real^7$ and our \textit{control} as $\bm{u} = \uB \in \real^3$. 


\paragraph{Discretization}

Without any (active) state constraints, we now that the optimal torque solution for the time-optimal reorientation is bang-bang. As such, we will elect to parameterize our control with either piecewise constant or piecewise linear basis functions. This amounts to discretizing our system using a zeroth order hold (ZOH) or a first order hold (FOH), respectively. 

First, we split the maneuver into $N$ evenly spaced temporal intervals. Since we will ultimately linearize the kinematics and dynamics, we want to choose $N$ such that over each interval this linearization remains valid. Given $\q(\ti)$ and $\q(\tf)$, we can compute the angle error between these according to
\begin{equation}
\varphi_{err} = 2*\cos^{-1} \left( \big[ \q(\tf)^* \otimes \q(\ti) \big]_{\text{s}} \right),
\label{eq:err_angle}
\end{equation}
where the operation $[\cdot]_{\text{s}}$ extracts the scalar part of a quaternion. 
% Assuming that a linearization is valid over $\zeta$ degrees, the number of discretization nodes is computed as 
% \begin{equation}
% N = \lceil \frac{\varphi_{err}}{\zeta} \rceil.
% \label{eq:N}
% \end{equation}

We now discretize time into $N$ evenly spaced intervals, where the size of each interval should be such that a linearization of our equations of motion remains roughly valid. For now we proceed with $N\in[5,10]$. 
\begin{equation}
\ti < t_1 < \ldots < t_{N-1} < \tf
\end{equation}
and represent the discretized time using $k \definedas t_k$. Now, instead of the \textit{continuous times} $t\in(\ti,\tf]$, we have the \textit{discrete times} $k\in\{\ti,\,t_1,\,\ldots,\,t_{N-1},\,\tf\}$.

I'm leaving out a bunch of details for now, but the discretization method leads to
\begin{equation}
\bm{X} = A \bm{X} + B \bm{U} + S s + \bm{R}
\label{eq:discrete_dynamics}
\end{equation}
where $\{\bm{A},\bm{B},\bm{S},\bm{R}\}$ are computed in the code by either \code{zoh.m} or \code{foh.m}. The constraints are all ``discretized'' by simply applying them at the discrete temporal nodes only. 

\paragraph{Virtual Control}

To guide the convergence process, we introduce a new term to address the issue of \textit{artificial infeasibility}. This stems from the fact that the linearization may produce an infeasible interation if there does not exists some admissible control that satisfies~\eqref{eq:discrete_dynamics}. We augment the dynamics with a virtual control term that allows us to synthetically retain feasibility according to
\begin{equation}
\bm{X} = A \bm{X} + B \bm{U} + \bm{S} s + \bm{R} + \bm{V}
\label{eq:VC_discrete_dynamics}
\end{equation}
where $\bm{V}$ can be thought of as an unconstrained input that needn't respect the physical dynamics of the problem. We heavily penalize the use of virtual control (so that we only use it when absolutely necessary) by augmenting the cost function to be $\tf + \wv \| \bm{V} \|_1$.

After these modifications, we now have a convex parameter optimization problem that can be run on-board the spacecraft. 

\begin{problem}\label{prob:prob2}
Find the final time $s$ and the control $\bm{U}\in\real^{3 \times N}$ such that
\begin{subequations}\label{eq:prob2}
\begin{align}
\min_{s,\bm{U},\bm{X},\bm{V}} &\quad s + \wv \| \bm{V} \|_1 \label{eq:p2_a}\\
\text{subject to:} &\quad \bm{X} = A \bm{X} + B \bm{U} + \bm{S} s + \bm{R} + \bm{V} \label{eq:p2_b}\\
&\quad \| \Hw \bm{x}_k \|_{\infty} \leq \wmax, \quad k=0,\ldots,N \label{eq:p2_c}\\ 
&\quad \| \bm{u}_k \|_{\infty} \leq \umax, \quad k=0,\ldots,\{N \text{ or } N-1\} \label{eq:p2_d}\\
&\quad \bm{x}_k^T \Hq^T \ME \Hq \bm{x}_k < 2, \quad k=0,\ldots,N \label{eq:p2_e}\\
&\quad \bm{x}_k^T \Hq^T \MI \Hq \bm{x}_k < 2, \quad k=0,\ldots,N \label{eq:p2_f}\\
&\quad \tfmin \leq s \leq \tfmax \label{eq:p2_g}\\
&\quad \bm{x}_0 = \big[ \q_0^T \; 0^T \big]^T, \; \bm{x}_N = \big[ \q_f^T \; 0^T \big]^T \label{eq:p2_h}
\end{align}
\end{subequations}
where $\bm{X} = \big[ \bm{x}_0^T \; \bm{x}_1^T \; \ldots \; \bm{x}_{N}^T ]^T \in \real^{7 \times N}$ and similarly for $\bm{U}$. Moreover, $\Hq = \big[ \eye{4} \; \zeros{4}{3} \big]$ and $\Hw = \big[ \zeros{3}{4} \; \eye{3} \big]$.
\end{problem}

\paragraph{Standard Form of Constraints}

Let us say that our combined \textit{solution vector} is 
\begin{equation*}
\bm{Z} = \begin{bmatrix}
\bm{X}^T & \bm{U}^T & s & \bm{V}^T & \bm{\eta}_v^T
\end{bmatrix} \in \real^{n_z}
\end{equation*}
where $n_z = N(3N_x + N_u)+1$. We will first map the cost function to standard form for second-order cone problems. The cost function must be linear, and so we will use the epigraph form to rewrite the cost function~\eqref{eq:p2_a} as
\begin{equation}
\bm{c}^T \bm{Z}, \quad \bm{c} = \big[ \zeros{1}{N n_x} \ \zeros{1}{N n_u} \ 1 \ \zeros{1}{N n_x} \ w_{\eta} \ones{1}{N n_x} \big]
\label{eq:std_cost}
\end{equation}
and add the following linear constraint to Problem~\ref{prob:prob2}
\begin{equation}
- \bm{\eta}_v \leq \bm{V} \leq \bm{\eta}_v.
\label{eq:p2_epi}
\end{equation}

We can thenwrite the equality constraints~\eqref{eq:p2_b} and~\eqref{eq:p2_h} as
\begin{equation}
\begin{bmatrix}
\big[ \eye{n_x} \ \zeros{n_x}{(N-1)n_x}] & \zeros{n_x}{N n_u} & \zeros{n_x}{1} & \zeros{n_x}{N n_x} & \zeros{n_x}{N n_x} \\
\bm{A} - \eye{N n_x} & \bm{B} & \bm{S} & \eye{N n_x} & \zeros{N n_x}{N n_x} \\
\big[ \zeros{n_x}{(N-1)n_x} \ \eye{n_x} \big] & \zeros{n_x}{N n_u} & \zeros{n_x}{1} & \zeros{n_x}{Nn_x} & \zeros{n_x}{N n_x}
\end{bmatrix} \bm{Z} = \begin{bmatrix}
\bm{x}_0 \\ -\bm{R} \\ \bm{x}_f
\end{bmatrix}.
\label{eq:std_equality}
\end{equation}
The linear inequality constraints in~\eqref{eq:p2_c}, \eqref{eq:p2_d}, \eqref{eq:p2_g} and \eqref{eq:p2_epi} are expressed as
\begin{equation}
\begin{bmatrix}
H_w^T \\ - H_w^T \\ H_u^T \\ - H_u^T \\ H_s^T \\ - H_s^T \\ H_v^T - H_{\eta_v}^T \\ -H_v - H_{\eta_v}^T
\end{bmatrix} \bm{Z} \leq \begin{bmatrix}
w_{\max} \ones{3N}{1} \\ w_{\max} \ones{3N}{1} \\ u_{\max} \ones{N n_u}{1} \\ u_{\max} \ones{N n_u}{1} \\ t_{f,\max} \\ -t_{f,\min} \\ \zeros{N n_x}{1} \\ \zeros{N n_x}{1}
\end{bmatrix},
\label{eq:std_ineq_lin}
\end{equation}
where,
\begin{gather*}
H_u = \begin{bmatrix}
\zeros{N n_x}{N n_u} \\ \eye{N n_u} \\ \zeros{2 N n_x + 1}{N n_u}
\end{bmatrix}, \quad H_s = \begin{bmatrix}
\zeros{N(n_x+n_u)}{1} \\ 1 \\ \zeros{2N n_x}{1}
\end{bmatrix}, \quad H_v = \begin{bmatrix}
\zeros{N (n_x+n_u) + 1}{N n_x} \\ \eye{N n_x} \\ \zeros{N n_x}{N n_x} 
\end{bmatrix}  \\
H_{\eta_v} = \begin{bmatrix}
\zeros{N(2n_x+n_u)+1}{N n_x} \\ \eye{N n_x}
\end{bmatrix}
\bar{H}_w = \begin{bmatrix}
\big[ \zeros{3}{4} \ \eye{3} \big] & \zeros{3}{n_x} & \cdots & \zeros{3}{n_x} \\
\vdots & \ddots & \ddots & \vdots \\
\zeros{3}{n_x} & \zeros{3}{n_x} & \cdots & \big[ \zeros{3}{4} \ \eye{3} \big]
\end{bmatrix} \quad H_w = \begin{bmatrix}
\bar{H}_w \\ \zeros{N (2n_x+n_u)+1}{3N} 
\end{bmatrix}
\end{gather*}
Without any quadratic constraints, we may now write Problem~\ref{prob:prob2} in standard form as
\begin{problem}
Find the vector $\bm{Z}\in\real^{n_z}$ such that
\begin{subequations}\label{eq:prob2}
\begin{align}
\min_{\bm{Z}} &\quad \bm{c}^T \bm{Z} \tag{\ref{eq:std_cost}}\\
\text{subject to:} &\quad A \bm{Z} = b \tag{\ref{eq:std_equality}} \\
&\quad G_{lin} \bm{Z} \leq h_{lin} \tag{\ref{eq:std_ineq_lin}}
\end{align}
\end{subequations}
\end{problem}


\end{document}